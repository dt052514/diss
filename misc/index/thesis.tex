% From https://github.com/UWIT-IAM/UWThesis

\documentclass [11pt, proquest] {uwthesis}[2015/03/03]

% fix for pandoc 1.14
\providecommand{\tightlist}{%
  \setlength{\itemsep}{0pt}\setlength{\parskip}{0pt}}

\newtheorem{theorem}{Jibberish}

%% \bibliography{references}

\hyphenation{mar-gin-al-ia}

%
% ----- apply watermark to every page
% ----- change 'stamp' to 'nostamp'
%------ to omit watermark
%
\usepackage[nostamp]{draftwatermark}
% % Use the following to make modification
\SetWatermarkText{DRAFT}
\SetWatermarkLightness{0.95}

%% for the per mil symbol
\usepackage[nointegrals]{wasysym}

%% for copyright symbol
\usepackage{textcomp}

%% to allow to rotate pages to landscape
\usepackage{lscape}
%% to adjust table column width
\usepackage{tabularx}

% suppress bottom page numbers on first page of each chapter
% because they overlap with text
\usepackage{etoolbox}
\patchcmd{\chapter}{plain}{empty}{}{}

%% for more attractive tables
\usepackage{booktabs}
\usepackage{longtable}


\usepackage{graphicx}


% Double spacing, if you want it.
% \def\dsp{\def\baselinestretch{2.0}\large\normalsize}
% \dsp

% If the Grad. Division insists that the first paragraph of a section
% be indented (like the others), then include this line:
% \usepackage{indentfirst}

%%%%%%%%%%%%%%%%%%
% If you want to use "sections" to partition your thesis
% un-comment the following:
%
% \counterwithout{section}{chapter}
% \setsecnumdepth{subsubsection}
% \def\sectionmark#1{\markboth{#1}{#1}}
% \def\subsectionmark#1{\markboth{#1}{#1}}
% \renewcommand{\thesection}{\arabic{section}}
% \renewcommand{\thesubsection}{\thesection.\arabic{subsection}}
% \makeatletter
% \let\l@subsection\l@section
% \let\l@section\l@chapter
% \makeatother
%
% \renewcommand{\thetable}{\arabic{table}}
% \renewcommand{\thefigure}{\arabic{figure}}
%
%%%%%%%%%%%%%%%%%%


%% Stuff from https://github.com/suchow/Dissertate

% The following line would print the thesis in a postscript font

% \usepackage{natbib}
% \def\bibpreamble{\protect\addcontentsline{toc}{chapter}{Bibliography}}

\setcounter{tocdepth}{1} % Print the chapter and sections to the toc
% controls depth of table of contents (toc): 0 = chapter, 1 = section, 2 = subsection

\usepackage[url=false,isbn=false,doi=false]{biblatex}

\prelimpages

%% from thesisdown
% To pass between YAML and LaTeX the dollar signs are added by CII
\Title{National Oil Companies as International Investors}
\Author{David J. Tingle}
\Year{2019}
\Program{Department of Government}
\Chair{Erik Voeten}{}{Department of Government}
\Signature{}
\Signature{}
\Signature{}

% commands and environments needed by pandoc snippets
% extracted from the output of `pandoc -s`
%% Make R markdown code chunks work
\usepackage{array}
\usepackage{amssymb,amsmath}
\usepackage{ifxetex,ifluatex}
\ifxetex
  \usepackage{fontspec,xltxtra,xunicode}
  \defaultfontfeatures{Mapping=tex-text,Scale=MatchLowercase}
\else
  \ifluatex
    \usepackage{fontspec}
    \defaultfontfeatures{Mapping=tex-text,Scale=MatchLowercase}
  \else
    \usepackage[utf8]{inputenc}
  \fi
\fi
\usepackage{color}
\usepackage{fancyvrb}
\DefineShortVerb[commandchars=\\\{\}]{\|}
\DefineVerbatimEnvironment{Highlighting}{Verbatim}{commandchars=\\\{\}}
% Add ',fontsize=\small' for more characters per line
\newenvironment{Shaded}{}{}
\newcommand{\KeywordTok}[1]{\textcolor[rgb]{0.00,0.44,0.13}{\textbf{{#1}}}}
\newcommand{\DataTypeTok}[1]{\textcolor[rgb]{0.56,0.13,0.00}{{#1}}}
\newcommand{\DecValTok}[1]{\textcolor[rgb]{0.25,0.63,0.44}{{#1}}}
\newcommand{\BaseNTok}[1]{\textcolor[rgb]{0.25,0.63,0.44}{{#1}}}
\newcommand{\FloatTok}[1]{\textcolor[rgb]{0.25,0.63,0.44}{{#1}}}
\newcommand{\CharTok}[1]{\textcolor[rgb]{0.25,0.44,0.63}{{#1}}}
\newcommand{\StringTok}[1]{\textcolor[rgb]{0.25,0.44,0.63}{{#1}}}
\newcommand{\CommentTok}[1]{\textcolor[rgb]{0.38,0.63,0.69}{\textit{{#1}}}}
\newcommand{\OtherTok}[1]{\textcolor[rgb]{0.00,0.44,0.13}{{#1}}}
\newcommand{\AlertTok}[1]{\textcolor[rgb]{1.00,0.00,0.00}{\textbf{{#1}}}}
\newcommand{\FunctionTok}[1]{\textcolor[rgb]{0.02,0.16,0.49}{{#1}}}
\newcommand{\RegionMarkerTok}[1]{{#1}}
\newcommand{\ErrorTok}[1]{\textcolor[rgb]{1.00,0.00,0.00}{\textbf{{#1}}}}
\newcommand{\NormalTok}[1]{{#1}}
\newcommand{\OperatorTok}[1]{\textcolor[rgb]{0.00,0.44,0.13}{\textbf{{#1}}}}
\newcommand{\BuiltInTok}[1]{\textcolor[rgb]{0.00,0.44,0.13}{\textbf{{#1}}}}
\newcommand{\ControlFlowTok}[1]{\textcolor[rgb]{0.00,0.44,0.13}{\textbf{{#1}}}}
\definecolor{gtownblue}{RGB}{1,30,65}
\definecolor{gtownblue2}{RGB}{0,34,105}
\definecolor{gtowngray}{RGB}{134,120,117}


\ifxetex
  \usepackage[setpagesize=false, % page size defined by xetex
              unicode=false, % unicode breaks when used with xetex
              xetex,
              colorlinks=true,
              linkcolor=gtownblue2]{hyperref}
\else
  \usepackage[unicode=true,
              colorlinks=true,
              linkcolor=gtownblue2]{hyperref}
\fi
\hypersetup{breaklinks=true, pdfborder={0 0 0}}
\setlength{\parindent}{0pt}
\setlength{\parskip}{6pt plus 2pt minus 1pt}
\setlength{\emergencystretch}{3em}  % prevent overfull lines
\setcounter{secnumdepth}{2} %% controls section numbering, e.g. 1 or 1.2, or 1.2.3


\begin{document}
\titlepage
\newpage
\copyrightpage

\setcounter{page}{-1}
\abstract{National Oil Companies (NOCs) were initially built to give states control over their domestic hydrocarbon resources, but many of them have now acquired assets and established subsidiaries outside of their home state. This dissertation adresses three key questions related to this phenomenon. First, how does NOC internationalization vary? Second, why are some NOCs highly internationalized and others aren't? Third, what are the political consequences of NOC internationalization? In addressing these questions, this research sheds light on a set of integral but under-theorized institutions, NOCs, which are at the centre of contemporary state capitalism and the political economy of hydrocarbon resource wealth.}

\acknowledgments{I have accumulated many debts in the course of researching and writing this dissertation.

Thanks are due, firstly, to Eric Voeten for his guidance, support, and encouragement throughout this process. I am also grateful to the many other mentors in the Georgetown community and beyond: these include in particular George Shambaugh, Nathan Jensen, and Paasha Mahdavi, but also Dan Nexon, Andrew Bennett, and Dan Hopkins.

I must also thank all of my colleagues who offered numerous helpful suggestions, stimulating conversations, and extensive support (intellectual, material, psychological). Chiefly -- but of course not exclusively -- Geoffrey Sigalet, Rebecca Lissner, Adam Mount, Mohit Grover, and Noel Anderson played important roles at key points in this project's development. I am very fortunate to have had you all as friends and colleagues.

My family has been extremely supportive throughout the process, whether or not they believed it would ever be finished. It wouldn't have been, except that I am blessed with a wonderful spouse. Kelly, your patience, encouragement, unflagging support throughout my time researching and writing made the difference. Thank you.}

\dedication{\begin{center}To Kelly, Emma, and Claire.\end{center}}

\tableofcontents
\listoffigures
\listoftables

\textpages

\begin{Shaded}
\begin{Highlighting}[]
\CommentTok{# To Render:}
\NormalTok{bookdown}\OperatorTok{::}\KeywordTok{render_book}\NormalTok{(}\StringTok{'index.Rmd'}\NormalTok{, huskydown}\OperatorTok{::}\KeywordTok{thesis_pdf}\NormalTok{(}\DataTypeTok{latex_engine =} \StringTok{'xelatex'}\NormalTok{))}
\end{Highlighting}
\end{Shaded}
\hypertarget{introduction-to-this-dissertation}{%
\chapter*{Introduction to this Dissertation}\label{introduction-to-this-dissertation}}
\addcontentsline{toc}{chapter}{Introduction to this Dissertation}

This section gives a 1,000 ft. overview of the dissertation as a whole. It has two key objectives. First, it is designed to provide a ``coles notes'' for the dissertation, clearly stating both the key questions and findings for all three substantive chapters. Second, should establish as clearly as possible how this dissertation addresses problems at the core of international political economy -- that is, it needs to establish disciplinary fit.

\hypertarget{paper1}{%
\chapter{What is NOC Internationalization?}\label{paper1}}

\hypertarget{intro01}{%
\section{Introduction}\label{intro01}}

In December of 2014, the Russian government engaged a drastic interest rate hike in the hopes of defending the ruble. There are a number of reasons for the Russian currency's precipitous decline in recent months --- these include the stock market cost of military adventurism in Ukraine and the general decline in oil prices. The drop in the ruble's value in mid-December, however, has been linked by observers to an ``opaque deal involving the central bank and the state-controlled oil company, Rosneft''" (Kramer, \protect\hyperlink{ref-kramer_russias_2014}{2014}). The oil company, \emph{The New York Times} goes on to report, ``had been clamoring for months for a government bailout to refinance debt the company ran up while making acquisitions when oil prices were high.''

\hypertarget{paper2}{%
\chapter{Why do some NOCs go global?}\label{paper2}}

\hypertarget{intro02}{%
\section{Introduction}\label{intro02}}

Nigeria and Angola are the two largest oil producers in Africa, and their NOCs (NNPC and Sonangol, respectively) are two of the most important domestic economic institutions. Both countries have significant levels of international investment into their oil sectors. Both experienced devastating civil conflict until early in the 21st century. Nigeria has a much larger population and economy; Angola performs better, comparatively speaking, on indexes that measure corruption and investment risk.

\hypertarget{colophon}{%
\chapter*{Colophon}\label{colophon}}
\addcontentsline{toc}{chapter}{Colophon}

This document is set in \href{https://github.com/georgd/EB-Garamond}{EB Garamond}, \href{https://github.com/adobe-fonts/source-code-pro/}{Source Code Pro} and \href{http://www.latofonts.com/lato-free-fonts/}{Lato}. The body text is set at 11pt with \(\familydefault\).

It was written in R Markdown and \(\LaTeX\), and rendered into PDF (and HTML) using \href{https://github.com/benmarwick/huskydown}{huskydown} and \href{https://github.com/rstudio/bookdown}{bookdown}. It was typeset using the XeTeX typesetting system. The TeX template used to ensure that this document meets the submission standards of the institution is heavily indebted to the Georgetown University \(\LaTeX\) template and the University of Washington \(\LaTeX\) template, and to Ben Marwick's heavy lifting in the \texttt{huskydown} package.

This version of the thesis was generated on 2020-06-25 01:53:08.

\backmatter

\hypertarget{references}{%
\chapter*{References}\label{references}}
\addcontentsline{toc}{chapter}{References}

\markboth{References}{References}

\noindent

\setlength{\parindent}{-0.20in}
\setlength{\leftskip}{0.20in}
\setlength{\parskip}{8pt}

\hypertarget{refs}{}
\leavevmode\hypertarget{ref-kramer_russias_2014}{}%
Kramer, A. E. (2014, December 16). Russia's Steep Rate Increase Fails to Stem Ruble's Decline. \emph{The New York Times}. Retrieved from \url{http://www.nytimes.com/2014/12/17/business/russia-ruble-interest-rates.html}
\end{document}
